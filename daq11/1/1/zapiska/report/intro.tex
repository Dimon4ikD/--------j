\section{Введение}

В данной курсовой работе рассматриваются модификации двух эталонных проектов <<Выпуклая оболочка>> и <<Изображение проекции полиэдра>>, реализованных на объектно-ориентированном языке программирования высокого уровня Ruby.

Целями работы являются:
\begin{itemize}
\item в проект <<Выпуклая оболочка>> добавить вычисление суммы внутренних углов выпуклой оболочки;
\item <<научить>> программу определять и выводить сумму углов, под которыми рёбра выпуклой оболочки пересекают заданную прямую в проекте <<Выпуклая оболочка>>;
\item добавить в проект <<Изображение проекции полиэдра>> вывод суммы площадей граней, все вершины которые расположены вне сферы $x^2+y^2+z^2=1$.
\end{itemize}

Для того чтобы выполнить полученные задания, необходимо было изучить особенности языка Ruby, подробно разобрать каждый эталонный проект и применить полученные знания в области информатики, компьютерной математики и аналитической геометрии на плоскости и в пространстве.

\endinput

