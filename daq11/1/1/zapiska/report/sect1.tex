\section{Модификация проекта <<Выпуклая оболочка>>}

\subsection{Постановка задачи}

Модифицируйте эталонный проект таким образом, чтобы вычислять сумму углов, под которыми рёбра выпуклой оболочки пересекаю заданную прямую.


\subsection{Решение}

После запуска программа предлагает пользователю ввести последовательно координаты вершин выпуклой оболочки. Введённая точка индуктивно добавляется в выпуклую оболочку. Нам же необходимо вместе со значениями периметра и площади выпуклой оболочки выводить сумму острых внутренних углов.
Это задание не требует серьёзной модификации исходного кода, все функции будут вписываться в файл проекта \verb|convex.rb|.

Для выполнения задания необходимо вычислять внутренние углы, используя координаты вершин выпуклой оболочки. В языке Ruby имеется особый механизм множественного наследования \verb|(mixin)| \verb|Math|, которая содержит методы вычисления простейших тригонометрических функций, в частности, в данном проекте используется \verb|Math.acos|.

Из аналитический геометрии извеcтно, что угол между векторами $\overline{BA}$ и $\overline{BC}$ можно найти по формуле 
$$\cos(B) =\dfrac{\overline{BA}\cdot \overline{BC}}{|\overline{BA}|\cdot|\overline{BC}|}.$$

Для точек $A(X_1;Y_1)$ и $B(X_2;Y_2)$ координаты вектора $${AB}=\{X_2-X_1;Y_2-Y_1\}.$$
Если $\overline{BA}\{X_1;Y_1\}$ и $\overline{BC}\{X_2;Y_2\}$, то скалярное произведение этих векторов
$$\overline{BA}\cdot \overline{BC}=X_1X_2+Y_1Y_2,$$ а длина вектора $$\sqrt{\overline{BA}^2}=|\overline{BA}|=\sqrt{X_1^2+Y_1^2}.$$

Окончательная формула для угла B между векторами $\overline{BA}$ и $\overline{BC}$ выглядит так:
$$\cos(B) =\dfrac{X_1X_2+Y_1Y_2}{\sqrt{X_1^2+Y_1^2}\cdot\sqrt{X_2^2+Y_2^2}}.$$

Метод \verb|Math.acos| возвращает угол в радианах. Так как решено выводить ответ в градусах, воспользуемся формулой $\alpha^\circ = \alpha \times \frac{180}{\pi}$.

\subsection{Модификация кода}



Метод, который находит уравнение прямой, заданной двумя точками:

\begin{small}
\begin{verbatim}
  def equation_from_segment(p1, p2)
 
    # Ax + By + C = 0
    x1, y1 = p1.x, p1.y
    x2, y2 = p2.x, p2.y

    a = y1 - y2
    b = x2 - x1
    c = x1 * y2 - x2 * y1
    [a, b, c]
  end
\end{verbatim}
\end{small}


Метод, который находит точку пересечения линия $AB$ и отрезка $PQ$:

\begin{small}
\begin{verbatim}
  def cross_point(a, b, p, q)
    a1, b1, c1 = equation_from_segment(a, b)
    a2, b2, c2 = equation_from_segment(p, q)
    d = (a1*b2-a2*b1)
    
    # Если прямые не перескаются - завершаем работу:
    # отрезки тоже не будут пересекаться:
    return false if d == 0

    # Находим координаты пересечения прямых:
    x = -(c1*b2-c2*b1).to_f / d
    y = -(a1*c2-a2*c1).to_f / d

    # Точка пересечения прямых должна принадлежать отрезку PQ:
    m = R2Point.new(x, y)
    return false unless m.inside?(p, q)
    m
  end 
\end{verbatim}
\end{small}

Метод нахождения угла $B$:

\begin{small}
\begin{verbatim}
  def angle (a, b, c)
    
    # Координаты векторов:
    ba_x = a.x - b.x
    ba_y = a.y - b.y
    bc_x = c.x - b.x
    bc_y = c.y - b.y

    dot_product = ba_x*bc_x + ba_y*bc_y 
    module_ba = ba_x**2 + ba_y**2
    module_bc = bc_x**2 + bc_y**2
    cos = dot_product / Math.sqrt(module_ba * module_bc)
    cos *= -1 if cos < 0
	    return Math.acos(cos)
  end
\end{verbatim}
\end{small}


Увидеть работу этой модификации можно на рис.~1.

\begin{figure}[ht!]
\begin{center}
\includegraphics[scale=0.6]{images/111}
\end{center}
\vspace*{-8mm}
\caption{Вычисление суммы углов, под которыми рёбра выпуклой оболочки пересекают заданную прямую}\label{fig:term_1}
\end{figure}
